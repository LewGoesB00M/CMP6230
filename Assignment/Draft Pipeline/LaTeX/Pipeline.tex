% -------------------------------------------------------------------------------
% Establish page structure & font.
\documentclass[12pt]{report}

\usepackage[total={6.5in, 9in},
	left=1in,
	right=1in,
	top=1in,
	bottom=1in,]{geometry} % Page structure

\usepackage{graphicx} % Required for inserting images
\graphicspath{{../../.images/}} % Any additional images I use (BCU logo, etc) are from here.

\usepackage[utf8]{inputenc} % UTF-8 encoding
\usepackage[T1]{fontenc} % T1 font
\usepackage{float}  % Allows for floats to be positioned using [H], which correctly
                    % positions them relative to their location within my LaTeX code.
\usepackage{subcaption}

% -------------------------------------------------------------------------------
% Declare biblatex with custom Harvard BCU styling for referencing.
\usepackage[
    useprefix=true,
    maxcitenames=3,
    maxbibnames=99,
    style=authoryear,
    dashed=false, 
    natbib=true,
    url=false,
    backend=biber
]{biblatex}

% Additional styling options to ensure Harvard referencing format.
\renewbibmacro*{volume+number+eid}{
    \printfield{volume}
    \setunit*{\addnbspace}
    \printfield{number}
    \setunit{\addcomma\space}
    \printfield{eid}}
\DeclareFieldFormat[article]{number}{\mkbibparens{#1}}

% Declare it as the bibliography source, to be called later via \printbibliography
\addbibresource{report.bib}

% -------------------------------------------------------------------------------
% To prevent "Chapter N" display for each chapter
\usepackage[compact]{titlesec}
\usepackage{wasysym}
\usepackage{import}

\titlespacing*{\chapter}{0pt}{-2cm}{0.5cm}
\titleformat{\chapter}[display]
{\normalfont\bfseries}{}{0pt}{\Huge}

% -------------------------------------------------------------------------------
% Custom macro to make an un-numbered footnote.

\newcommand\blfootnote[1]{
    \begingroup
    \renewcommand\thefootnote{}\footnote{#1}
    \addtocounter{footnote}{-1}
    \endgroup
}

% -------------------------------------------------------------------------------
% Fancy headers; used to show my name, BCU logo and current chapter for the page.
\usepackage{fancyhdr}
\usepackage{calc}
\pagestyle{fancy}

\setlength\headheight{37pt} % Set custom header height to fit the image.

\renewcommand{\chaptermark}[1]{%
    \markboth{#1}{}} % Include chapter name.


% Lewis Higgins - ID 22133848           [BCU LOGO]                [CHAPTER NAME]
\lhead{Lewis Higgins - ID 22133848~~~~~~~~~~~~~~~\includegraphics[width=1.75cm]{BCU}}
\fancyhead[R]{\leftmark}

% ------------------------------------------------------------------------------
% Used to add PDF hyperlinks for figures and the contents page.

\usepackage{hyperref}

\hypersetup{
    colorlinks=true,
    linkcolor=black,
    filecolor=magenta,
    urlcolor=blue,
    citecolor=black,
}

% ------------------------------------------------------------------------------
\usepackage{xcolor} 
\usepackage{colortbl}
\usepackage{longtable}
\usepackage{amssymb}
% ------------------------------------------------------------------------------


% -------------------------------------------------------------------------------

\title{CMP6230 Draft Pipeline}
\author{Lewis Higgins - Student ID 22133848}
\date{November 2024}

% -------------------------------------------------------------------------------

\begin{document}


\makeatletter
\begin{titlepage}
    \begin{center}
        \includegraphics[width=0.7\linewidth]{BCU}\\[4ex]
        {\large \bfseries  \@title }\\[2ex]
        {\large \bfseries  DRAFT VERSION }\\[2ex]
        {\@author}\\[30ex]
        {Word count: XXXX}\\[20ex]
    \end{center}
\end{titlepage}
\makeatother
\thispagestyle{empty}
\newpage


% Page counter trick so that the contents page doesn't increment it.
\setcounter{page}{0}

\tableofcontents
\thispagestyle{empty}


\chapter{Candidate Data Sources}
For the first stage of the pipeline, data ingestion, three data sources will be identified in order to find 
the one that would be most optimal for the production and deployment of a machine learning model to complete 
a supervised learning task.

\section{Notes - DELETE BEFORE SUBMISSION}
Lucidchart can generate ERDs from CSVs.
For each, show the pandas head, column data types and ERD.
Finding one with multiple CSVs can be good to make the ERD look more complex. 
\textbf{So far, all of your data is from Kaggle. Consider another source like data.gov.uk, especially
considering that it can give you raw data for preprocessing.} On Kaggle, the "Provenance" section will 
have the source if the description doesn't. If neither have a source, it's probably fake data.

\begin{itemize}
    \item Loan data \begin{itemize}
        \item Fictional, so hard to give a good problem statement because this isn't real.
        \item Classification - Should they be given a loan?
        \item Unlikely to be of any use, should find another.
    \end{itemize}
    \item Smoke detection \begin{itemize}
        \item Real data
        \item Lots to explain (how the alarms work etc)
        \item Preprocessed, but you could still do more (remove timestamp etc)
        \item Classification - Should the smoke/fire alarm sound?
    \end{itemize}
    \item Employee data \begin{itemize}
        \item Allegedly real data though it seems hard to believe.
        \item Classification - Is the employee likely to find another job instead?
    \end{itemize}
    \item Australian weather \begin{itemize}
        \item Promising. Real data, but very large. Can do lots of preprocessing.
        \item Classification - Will it rain tomorrow?
    \end{itemize}
    \item Cardiovascular disease \begin{itemize}
        \item Good data, enormous amount of it (laptop might not handle it), good source. 
        \item However, it's already been processed. Check if that's fine or not.
        \item Classification - Do they have heart disease?
    \end{itemize}
    \item Diabetes \begin{itemize}
        \item In consideration for CMP6200, cannot use a dataset in both.
        \item It's actually real data according to the Kaggle page, from "multiple healthcare providers" and the 
        electronic health records (EHRs) they keep.
        \item Preprocessed already, but duplicates exist in it.
        \item Only 9 features, is that a bad thing?
        \item Classification - Do they have diabetes?
    \end{itemize}
\end{itemize}

\noindent \large \textbf{
The size of the datasets used for this module do not matter, unlike CMP6202.
As such, OpenML and the smaller Kaggle sets are allowed. It might actually be 
\textit{disadvantageous} for you to use large datasets because this will be done on a VM.} 
Consider the classic Heart dataset with the 300 values. Simple and instant processing, 
even on your laptop. Even if you don't use it, it should be a candidate. 

\section{Candidate 1 - Smoke Detection Dataset}



\end{document}