\section{Initial Software Setup}
This section details the configuration of the key software and packages within the pipeline. The installation 
processes of Miniconda and Docker are very lengthy and can therefore be found in Appendix A.

\subsection{VirtualBox \& Ubuntu}
The majority of data scientists utilise Linux distributions for their projects due to its open-source nature and the 
availability of many tools and packages. Therefore, VirtualBox, software to create a virtual machine which runs inside the host 
machine \autocite{oracle_oracle_nodate}, will be used to virtualise an Ubuntu 22.04 LTS system on the Windows host machine. This 
particular OS was chosen because of its relative recency and frequent software \& security updates. Virtual machines
also allow for "snapshots", which store the current state of the machine and its files at that time to be restored at any point in the 
event of unforeseen errors. Should a catastrophic error that would damage the system occur, the host machine would be unaffected as the 
virtual machine is isolated.

\begin{figure}[H]
    \centering
    \includegraphics[width=.5\linewidth]{Implementation/VBoxConfig.png}
    \caption{The configuration for the pipeline's VM.}
    \label{fig:VBoxConfig}
\end{figure}

\begin{figure}[H]
    \centering
    \includegraphics[width=.75\linewidth]{Implementation/Neofetch.png}
    \caption{The VM's Neofetch display.}
    \label{fig:Neofetch}
\end{figure}

\subsection{Conda environment and packages}
This section details the installation of packages to the Pipeline environment.

\begin{figure}[H]
    \centering
    \includegraphics[width=.75\linewidth]{Implementation/Conda/CondaCreation.png}
    \caption{Creating the "Pipeline" Conda environment.}
    \label{fig:CondaCreation}
\end{figure}

\para Python 3.8 is used due to package compatibility issues with later versions of Python.

\begin{figure}[H]
    \centering
    \includegraphics[width=\linewidth]{Implementation/Conda/CondaPackages.png}
    \caption{Installing packages to the environment via Conda.}
    \label{fig:CondaPackages}
\end{figure}

\subsection{Docker and Docker images}
This section covers the downloading of the two necessary Docker images, MariaDB Columnstore and Redis. 
Their usage is further elaborated upon later in this chapter. The images for the containers will first need to be
downloaded from Docker's repository, shown in Figure \ref{fig:DockerPull}.

\begin{figure}[H]
    \centering
    \includegraphics[width=\linewidth]{Implementation/Docker/Containers/Pull.png}
    \caption{Pulling the Docker images.}
    \label{fig:DockerPull}
\end{figure}

\pagebreak 
\subsubsection{MariaDB Columnstore}
MariaDB Columnstore runs on port 3306, but is accessed through port 3307.

\begin{figure}[H]
    \centering
    \includegraphics[width=\linewidth]{Implementation/Docker/Containers/MariaDB/1.png}
    \caption{Creating the MariaDB Columnstore container.}
    \label{fig:CreateMCS}
\end{figure}

\begin{figure}[H]
    \centering
    \includegraphics[width=\linewidth]{Implementation/Docker/Containers/MariaDB/2.png}
    \caption{Creating the user account for MariaDB.}
    \label{fig:CreateMCSUser}
\end{figure}

\begin{figure}[H]
    \centering
    \includegraphics[width=\linewidth]{Implementation/Docker/Containers/MariaDB/3.png}
    \caption{Creating the database for later use.}
    \label{fig:CreateDB}
\end{figure}

\pagebreak 
\subsubsection{Redis}
Redis runs on port 6379.

\begin{figure}[H]
    \centering
    \includegraphics[width=\linewidth]{Implementation/Docker/Containers/Redis/1.png}
    \caption{Creating the Redis container.}
    \label{fig:CreateRedis}
\end{figure}


\subsection{Airflow and MLFlow initialisation}
Airflow and MLFlow do not immediately work upon install, and must first be initialised.

\subsubsection{Airflow}
\begin{figure}[H]
    \centering
    \includegraphics[width=\linewidth]{Implementation/Airflow/Initialisation/1.png}
    \caption{Initialising Airflow's database.}
    \label{fig:AirflowInit}
\end{figure}

\begin{figure}[H]
    \centering
    \includegraphics[width=\linewidth]{Implementation/Airflow/Initialisation/2.png}
    \caption{Creating an administrative Airflow user.}
    \label{fig:AirflowUser1}
\end{figure}

\begin{figure}[H]
    \centering
    \includegraphics[width=\linewidth]{Implementation/Airflow/Initialisation/3.png}
    \caption{Verifying that the user was created.}
    \label{fig:AirflowUser2}
\end{figure}

\begin{figure}[H]
    \centering
    \includegraphics[width=\linewidth]{Implementation/Airflow/Initialisation/4.png}
    \caption{Successfully starting Airflow's web server.}
    \label{fig:AirflowWebserver}
\end{figure}

\begin{figure}[H]
    \centering
    \includegraphics[width=\linewidth]{Implementation/Airflow/Initialisation/5.png}
    \caption{Successfully starting Airflow's task scheduler.}
    \label{fig:AirflowScheduler}
\end{figure}

\subsubsection{MLFlow}

\begin{figure}[H]
    \centering
    \includegraphics[width=\linewidth]{Implementation/MLFlow/Initialisation/1.png}
    \caption{Creating a directory for MLFlow and initialising its database.}
    \label{fig:MLFlowInit}
\end{figure}

\begin{figure}[H]
    \centering
    \includegraphics[width=\linewidth]{Implementation/MLFlow/Initialisation/3.png}
    \caption{Running MLFlow's frontend web UI.}
    \label{fig:MLFlowUICmd}
\end{figure}

\begin{figure}[H]
    \centering
    \includegraphics[width=\linewidth]{Implementation/MLFlow/Initialisation/2.png}
    \caption{MLFlow's web UI.}
    \label{fig:MLFlowEmptyUI}
\end{figure}
